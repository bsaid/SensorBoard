\chapter{Conclusion}
The work was finally split into four milestones. The first milestone was achieved when I have decided to develop a new data logging hardware, because the existing solutions didn't fulfilled all the requirements. I have successfully developed the SensorBoard hardware in the second step and then the board was manufactured in six copies. I have achieved the second milestone with the manufactured hardware. The third phase was about testing the hardware and about implementing libraries for communication between the controller and other parts, mainly sensors. I have found several mistakes in the hardware design during programming, but all of them were corrected. The third milestone was achieved when I had a working hardware with C++ libraries for easier communication with sensors and other parts of the hardware. These libraries are a part of the API. During the last phase I have implemented the firmware for real-time movement analysis. When the SensorBoard is placed on a horse under the saddle, it is able to recognize the kind of movement of a horse. This firmware demonstrates the functionality of the designed hardware in real use. I have achieved the last milestone during the first successful test of the platform on a horse's body.

There are many applications and use cases of the SensorBoard presented as examples in the thesis. I have selected the movement analysis case, because this task started the whole project in the begining. The movement analysis firmware shows the advantages of running some user code on the hardware. We can see the results of the analysis in real-time.

If there is an interest on future work with this hardware, I would recommend to create the second version of the SensorBoard with attention to the list of hardware errors presented in this thesis. The actual version was designed as a prototype and it's not recommended for production. The prototype of the SensorBoard was used for example as a simple autopilot of a small quadcopter, so I believe that it will be used in several applications in the near future. I hope that the firmware for the movement analysis of a horse will help to work with this kind of animals easier.