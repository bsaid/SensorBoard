\chapter{Introduction}
Currently, the inertial sensors are more widely used, and their price is decreasing. The decreasing price allows creating new solutions with lower costs. Some examples can be found in wearable devices, mobile phones, navigation or control systems. There are several devices for various use cases on the market.

These devices usually cover their use cases and do not have any additional versatility. For example, some wearable devices require permanent connection and do not have enough memory to log the captured data for several minutes. Another hardware needs an external power supply and cannot work independently. Some next boards are closed and do not allow to run user code. The remaining electronics that fulfills all the conditions above is very expensive.

The goal of this thesis is to develop a new independent device with several sensors and a user programmable controller. So, why we do not use a mobile phone? Well, the device should be able to run a real-time user program and to read the sensors at a precisely specified frequency. The size and weight of the hardware are also critical. With the decreasing price of MEMS (Micro Electro Mechanical Systems) sensors and controllers, we can build cheaper and lighter device than a smartphone.

In this thesis, I have utilized a microcontroller, which is new on the market. This microcontroller was released on September 2016, and it is designed for various real-time embedded tasks, and it is manufactured with on-chip WiFi and Bluetooth.

This paper goes through the development process of the new wearable hardware including manufacturing, testing, API implementation, firmware for logging data to SD card and the first use of the prototype outside the laboratory. Each part is discussed in the separated chapter.

The main motivation was to create an independent wearable device with various sensors and enough memory for storing the measured data. The capacity of the battery and power consumption is also significant. During the design process, I have added some additional requirements that do not increase the cost very much and give us higher versatility and possibility to use the new hardware in more solutions.

When I finished the development process of the hardware, I had to prove the announced functionality of the device. I have selected one task about movement analysis of a horse using inertial sensors. The analysis of human movement is a widely studied problem. No other animal is so widely used in sports activities. For example, horses are the only animals participated in Olympics Games. When the device is mounted on a horse's body, the real-time analysis is able to detect the type of movement of this horse. With this task, I can compare the functionality of my device to the other solutions available on the market. The movement analysis task is based on outdoor measurements so we can see the usability of my hardware and software outside of the laboratory conditions. These successful tests are mentioned in another part of this thesis.

When I was looking for a suitable algorithm for determination of the movement based on data from inertial sensors, I wrote my own algorithm with low requirements on the CPU power. The algorithm runs real-time with latency around one step of the horse. Lower latency cannot be achieved because one step is the basic element of each movement. During the outdoor tests, we could see the horse with the SensorBoard mounted on its back and the real-time results of the analysis on the screen of a connected tablet.

The algorithm first determines fundamental types of movements and then it is looking for more advanced ones based on previous results. The types of movement are defined as statements, so the new types can be detected without modification of the computation code. These tests achieved the last milestone of this thesis.
