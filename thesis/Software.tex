\chapter{Software}

//todo: o jakem HW mluvime
- programuje se ESP a BMF
- periferie, ktere jsou k dispozici + schematicke zapojeni procesoru a periferii

\begin{figure}
	\centering
	\label{fig:SWmodules}
	\caption{Schema of the modules available for user program}
	%\includegraphics[keyvals]{imagefile}
\end{figure}

\section{Programming the Board}
//todo: dva procesory, ktery je ktery

\subsection{Programming ESP32}
//todo: ktery to je, ze je na uzivatelsky program nebo na hlavni program, pristup k periferiim, odkaz na datasheet, zaklad je ESP-IDF framework, zbytek je rozsireni

\subsubsection{ESP-IDF Framework}
//todo: POSIX kompatibilita

\subsubsection{Arduino Compatibility}

\subsubsection{MicroPython Compatibility}

\subsubsection{Matlab and Simulink}
//todo: ?

\subsection{Programming BMF055}
//todo: jde o Atmel ARM, ktery, co vsechno v tom BMF055 cipu je, odkaz na datasheet

\subsubsection{Atmel Studio Framework}
//todo: jiny nazev?

\subsubsection{MicroPython Compatibility}
//todo: ?

\subsubsection{Matlab and Simulink}
//todo: ?

\section{Hardware Access Layer}
- ke kazde periferii popis

\subsection{ESP32}

\subsubsection{Software Tools}
- SW vlastnosti jako POSIX kompatibilita, FreeRTOS atd.

\subsection{BMF055}

\subsubsection{Software Tools}

\section{Usage Examples}
- priklady naprogramovani - logovani jen ESP, sensorfusion na BMF, navigace na ESP a vypocty na BMF atd.

\section{Data Logging Firmware}
- SW pro analyzu namerenych dat
- jak funguje sbirani dat, synchronizace jednotlivych casti atd.
