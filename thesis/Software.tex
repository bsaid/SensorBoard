\chapter{Software}

//todo: o jakem HW mluvime
- programuje se ESP a BMF

\begin{figure}
	\centering
	\label{fig:SWmodules}
	\caption{Schema of the modules available for user program}
	%\includegraphics[keyvals]{imagefile}
\end{figure}

\section{Programming the board}
//todo: dva procesory, ktery je ktery

\subsection{Programming ESP32}
//todo: ktery to je, ze je na uzivatelsky program nebo na hlavni program, pristup k periferiim, odkaz na datasheet

\subsubsection{ESP-IDF framework}

\subsubsection{Arduino compatibility}

\subsubsection{MicroPython compatibility}

\subsection{Programming BMF055}
//todo: jde o Atmel ARM, ktery, co vsechno v tom BMF055 cipu je, odkaz na datasheet

- zpusoby naprogramovani, ESP-IDF, Arduino, Python, AtmelStudio
- periferie, ktere jsou k dispozici + schematicke zapojeni procesoru a periferii
- k periferiim zminka o vlastnostech jako POSIX kompatibilita, FreeRTOS atd.
- priklady naprogramovani - logovani jen ESP, sensorfusion na BMF, navigace na ESP a vypocty na BMF atd.

- HAL na ESP a mozna na BMF

- SW pro analyzu namerenych dat
- jak funguje sbirani dat, synchronizace jednotlivych casti atd.
