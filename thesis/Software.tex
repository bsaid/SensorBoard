\chapter{Software}
The Sensor Board hardware has two programmable CPUs and several configurable chips. The main ESP32 CPU (two low-power Xtensa 32-bit LX6 microprocessors) \cite{espressif:ESP-WROOM-32} is programmable via USB or Bluetooth or JTAG. The JTAG connector is not present on the Sensor Board. The second microcontroller is a part of BMF055 \cite{bosch:BMF055} multifunctional chip. It is Atmel SAMD20 \cite{atmel:SAMD20} with ARM Cortex-M0+ CPU programmable via SWD interface \cite{SWDinterface}.

\paragraph{Microcontrollers:}
\begin{enumerate}
	\item Espressif ESP-WROOM-32 \cite{espressif:ESP-WROOM-32}:
		\begin{itemize}
			\item dual core, 240 MHz, 448 kB ROM, 520 kB SRAM, 4 MB SPI flash memory
			\item Designated for main program handling all communication and interaction with user or other devices.
		\end{itemize}
	\item Atmel SAMD20 \cite{atmel:SAMD20}:
		\begin{itemize}
			\item ARM Cortex-M0+ CPU, 48 MHz, 32 kB SRAM, 256 kB flash memory
			\item Designated for processing inertial data and computing sensor fusion, can be used for example as an emulator of other sensors or as a simple flight controller.
		\end{itemize}
\end{enumerate}

//todo: periferie, ktere jsou k dispozici + schematicke zapojeni procesoru a periferii
\begin{figure}
	\centering
	\label{fig:SWmodules}
	\caption{Schema of the modules available for user program}
	%\includegraphics[keyvals]{imagefile}
\end{figure}

\section{Programming the Board}
Each processor on the Sensor Board has to be programmed separately via its interface. Only ESP32 \cite{espressif:ESP-WROOM-32} (main processor) can be programmed over the air via Bluetooth. This feature is disabled by default.

\subsection{Programming ESP32}
There are several ways how to program and use the ESP32 \cite{espressif:ESP-WROOM-32} controller. The official framework is Espressif IoT Development Framework (ESP-IDF) \cite{ESP-IDF} and supports all the chip functionality.

The chip can be programmed using Arduino compatible framework which creates an easy way for prototyping and 

//todo: ktery to je, ze je na uzivatelsky program nebo na hlavni program, pristup k periferiim, odkaz na datasheet, zaklad je ESP-IDF framework, zbytek je rozsireni

\subsubsection{ESP-IDF Framework}
//todo: POSIX kompatibilita

\subsubsection{Arduino Compatibility}

\subsubsection{MicroPython Compatibility}

\subsubsection{Matlab and Simulink}
//todo: ?

\subsection{Programming BMF055}
//todo: jde o Atmel ARM, ktery, co vsechno v tom BMF055 cipu je, odkaz na datasheet

\subsubsection{Atmel Studio Framework}
//todo: jiny nazev?

\subsubsection{MicroPython Compatibility}
//todo: ?

\subsubsection{Matlab and Simulink}
//todo: ?

\section{Hardware Access Layer}
- ke kazde periferii popis

\subsection{ESP32}

\subsubsection{Software Tools}
- SW vlastnosti jako POSIX kompatibilita, FreeRTOS atd.

\subsection{BMF055}

\subsubsection{Software Tools}

\section{Usage Examples}
- priklady naprogramovani - logovani jen ESP, sensorfusion na BMF, navigace na ESP a vypocty na BMF atd.

\section{Data Logging Firmware}
- SW pro analyzu namerenych dat
- jak funguje sbirani dat, synchronizace jednotlivych casti atd.
