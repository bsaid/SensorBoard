\chapter{Introduction}
Attitude sensors are more widely used and their price is decreasing. The decreasing price allows creating new solutions with lower costs. Some examples can be found in wearable devices, mobile phones, navigation or control systems. There are several devices for various use cases on the market.

These devices usually cover their use cases and don't have any additional versatility. For example, some wearable devices require permanent connection and don't have enough memory to log the captured data for several minutes. Another hardware requires an external power supply and cannot work independently. Some next boards are closed and don't allow to run a user code. The remaining hardware that fulfills all the conditions above is usually very expensive.

The goal of this work is to develop a new independent device with several sensors and a user programmable controller. So, why we don't use a mobile phone? Well, the device should be able to run a real-time user program and to read the sensors at a precisely specified frequency. The size and weight of the device are also very important. With the decreasing price of \ac{MEMS} sensors and controllers, we are able to build cheaper and lighter device than a smartphone.

I have utilized a new microcontroller in this work. This microcontroller was released on September 2016 and it's designed for various real-time embedded tasks and it's manufactured with on-chip WiFi and Bluetooth. I think that at least one wearable device should be designed with this microcontroller.

This paper goes through the development process of the new wearable hardware including manufacturing, testing, API implementation, firmware implementation and the first use of the prototype outside the laboratory. Each part is discussed in the separated chapter.

The main motivation was to create an independent wearable device with various sensors and enough memory for storing the measured data. The capacity of the battery and power consumption is also very important. During the design process, I have added some additional requirements that don't increase the cost very much, but they give us higher versatility and possibility to use the new hardware in more solutions.

When I finished the development process of the hardware I had to prove the announced functionality of the device. I have selected a task about movement analysis of a horse using inertial sensors. When the device is mounted on a horse's body, the real-time analysis is able to detect the type of movement of this horse. With this task, I can compare the functionality of my device to the other hardware available on the market. The movement analysis task is based on outdoor measurements, so I can see the usability of my hardware and software outside laboratory conditions. These successful tests are also part of this thesis.

When I was looking for a good algorithm for determination of the movement based on data from inertial sensors, I tried to write my own algorithm with low requirements on the CPU power. The algorithm runs real-time with latency around one step of the horse. Lower latency cannot be achieved because one step is the basic element of each movement. During the outdoor tests, we could see the horse with the SensorBoard mounted on its back and the real-time results of the analysis on the screen of a connected tablet.

The algorithm first determines very basic types of movements and then it's looking for more advanced ones based on previous results. The types of movement are defined as statements, so new types of movement of any subject can be detected without modification of the computation code. These tests achieved the last milestone of this thesis.
